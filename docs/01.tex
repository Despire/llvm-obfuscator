\section{Computer Security}

When we talk about computer security, we most often talk about how to secure our data and communications from being read by a malicious actor.
This could be achieved by having a secret key shared between two parties that the third party does not know, this branch of security is known as cryptography.

Computer security is a vast topic in itself and has many branches that focus on different areas.
While cryptography addresses the problem of secure communication and encryption of data, 
it is based on the fundamental assumption that the secret key is securely stored.

Consider this scenario, you have a software product packaged in a binary and you decide to distribute it to 
your customers who paid for it. These binaries may contain secrets that you don't want anyone else
to have except the customer who paid for your product. These secrets may be keys, unique identification
numbers or anything that you consider confidential. These secrets that are distributed with the binary
are then used when communicating with you.

What happens when a 3rd party gets hold of one or more of the binaries and decide to re-distribute
them, perhaps with a cheaper price, some changes to the features etc. The malicious actor can do whatever
he wants as there is no limit as to what can and can't be done when he gets hold of the binary. This is 
where Surreptitious software can be useful.

\subsection{Surreptitious software}

Surreptitious software is a term that describes a branch of computer security that works with techniques
from many branches of computer security, namely cryptography, watermarking, reverse engineering, steganography
and compilers \cite{serr-soft}.

You may be wondering how we could have avoided the scenario just described using these techniques.
Watermarking could be used to track illegal copies, tamperproofing could be used to prevent tampering with the binary,
obfuscation could be used to protect algorithms.

As with cryptography, where the secret keys used have different lifetimes and will not prevent a third party from reading the data forever,
Surreptitious Software does not guarantee that these techniques will work forever or be broken. 
The goal is to slow down the malicious actor so that breaking these algorithms takes more time than they are willing to invest and potentially give up \cite{serr-soft}.

You may be asking yourself why would you want to store secrets in your binary that could be exploited. There is no
judgement to this question, whatever the case is its good to know what can be used to improve the security when such
a use cases arises.

\subsubsection{Examples of Surreptitious software}

A quick glance at the following website https://patents.google.com/ with keywords, watermarking, obfuscation, tamperproofing etc... will tell us which companies use Surreptitious software.
For example, Microsoft Inc. owns several patents related to obfuscation, watermarking \cite{ms-patent01},
Apple Inc. owns patents on obfuscation \cite{apple-patent01}, and there are many more such examples.

These are only those that are public publicly accessbile, The use-cases for Surreptitious software can range from simply
protecting Games from being cracked to protecting Military equipment \cite{serr-soft}.






